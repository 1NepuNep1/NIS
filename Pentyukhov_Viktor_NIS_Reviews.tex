\documentclass[14pt]{article}

\usepackage[utf8]{inputenc} % allow utf-8 input
\usepackage[russian]{babel}
\usepackage{tempora} %Times New Roman alike
\linespread{1.5}
\usepackage[T1]{fontenc}    % use 8-bit T1 fonts
\usepackage[colorlinks=true, linkcolor=black, citecolor=blue, urlcolor=blue]{hyperref}       % hyperlinks
\usepackage[left=1.5cm,right=1.5cm,top=2cm,bottom=2cm]{geometry}
\usepackage{url}            % simple URL typesetting
\usepackage{booktabs}       % professional-quality tables
\usepackage{amsfonts}       % blackboard math symbols
\usepackage{nicefrac}       % compact symbols for 1/2, etc.
\usepackage{microtype}      % microtypography
\usepackage{lipsum}  % Can be removed after putting your text content
\usepackage{graphicx}
\usepackage{natbib}
\usepackage{doi}
\setcitestyle{aysep={,}}



\title{ \vspace{230px} \textbf{Применение искусственного интеллекта (ИИ) в разработке видеоигр}}

\author{Пентюхов Виктор Ильич\\
\AND
Группа: БЭАД223\\
\AND
\AND
\AND
\AND
\AND
}

% Uncomment to remove the date
\date{December 2021}



\begin{document}

\maketitle


\newpage
\tableofcontents
\thispagestyle{empty}



\newpage
\setcounter{page}{1}
\section{Введение}
В современном мире индустрия компьютерных игр занимает огромную нишу как в сфере развлечения, так и в сфере экономики и бизнеса, становясь одной из ключевых отраслей цифровой экономики. С постоянным стремительным развитием технологий в последние десятилетия, искусственный интеллект (ИИ) стал неотъемлемым компонентом инновационных изменений в данной области, перерабатывая привычные стандарты и взращивая новые горизонты творчества.

Применение искусственного интеллекта в разработке видеоигр привносит не только новые уровни реализма и взаимодействия, но и существенно влияет на сам игровой процесс, создавая уникальные и захватывающие виртуальные миры, в которых игроки могут полностью погрузиться. Этот взаимодействующий мир воплощается благодаря продвинутым алгоритмам ИИ, способным адаптироваться к действиям игроков, предугадывать их поведение и динамически реагировать на изменяющиеся сценарии.

В данном обзоре мы не только рассмотрим ключевые аспекты использования искусственного интеллекта в разработке видеоигр, но и глубоко проанализируем его влияние на геймдизайн. Мы выявим, как ИИ перекраивает традиционные грани между виртуальным и реальным, придавая игровым персонажам не только выразительность, но и способность к обучению и эволюции в процессе игры.

Также мы рассмотрим, как искусственный интеллект революционизирует процесс создания визуальных элементов в играх. Генерация картинок и видеороликов с применением ИИ открывает новые перспективы для дизайнеров и разработчиков, позволяя автоматизировать и усовершенствовать процессы создания графики и анимации. Эта технология не только снижает трудозатраты на создание визуальных эффектов, но и позволяет создавать более детализированные и удивительные игровые миры.

Оценим преимущества, вызовы и перспективы данного инновационного подхода, рассмотрим, как он формирует новые стандарты развлечений и какие возможности открываются для разработчиков и игроков в эпоху все более продвинутого искусственного интеллекта.


\section{Обзоры Статей}
\label{sec:headings}

\subsection{«The Role of Artificial Intelligence in Video Game Development», Aleksandar Filipović}
\subsubsection{Краткое описание статьи}
В данной статье подробно рассматривается роль ИИ в разработке компьютерных игр, с акцентом на различные аспекты применения ИИ в этой отрасли. Во введении автор рассматривает развитие индустрии видеоигр и ИИ, определяя развитие роли ИИ в видеоиграх. В следующем разделе статьи автор рассматривает основные концепции, такие как Не-Игровые Персонажи (NPC) и как ИИ повышает их интеллект в игре. Он объясняет, как различные методы ИИ используются для принятия решений, отслеживания игроков и адаптации игры к их действиям. Более того, помимо NPC Александер затрагивает такие темы, как процедурно генерируемые миры и адаптивная генерация уровней, так как такое применение ИИ способствует более глубокому и более динамичному опыту игрока. В заключение автор рассматривает современные методы применения ИИ для персонализации игр, анализа пользовательских данных и улучшения графики и звука. Такие методы требуют более комплексного подхода, в них используется как машинное обучение, так и разработка глубоких нейронных сетей.

\subsubsection{Роль статьи в обзоре}
 Статья, написанная данным профессором, очень важна, так как в ней автор по сути рассматривает почти все возможные приемы использования ИИ в разработке видеоигр (за исключением генерации картинок загрузочных экранов, видеороликов и 3D-Моделей). Он рассматривает основные концепты и алгоритмы, которые используют в каждом из этих методов, поясняет их работу. Как итог: статья была бы полезна как неразбирающимся в теме пользователям, так и людям, имеющим в ней определенный опыт, так как она содержит в себе широкий спектр информации о разработке видеоигр с использованием ИИ, который может быть не только полезен, но и безумно интересен.

\subsection{«Knowledge acquisition for adaptive game AI», Marc Ponsen, Pieter Spronck, Héctor Muñoz-Avila, David W. Aha }
\subsubsection{Краткое описание статьи}
В данной статье описывается концепт «Адаптивного Игрового ИИ». Авторы рассказывают, что это система, способная изменять свои стратегии и поведение в соответствии с динамикой игровой ситуации. Основной целью адаптивного игрового ИИ является создание более реалистичного и интересного опыта игры. Этот вид ИИ, по словам авторов статьи, способен анализировать действия игрока, принимать решения и адаптировать свои тактики для оптимизации игрового процесса. Он может использовать различные методы, такие как машинное обучение, генетические алгоритмы или усиление обучения, чтобы улучшать свои навыки и приспосабливаться к уровню мастерства игрока. Авторы не только описали сам концепт Адаптивного Игрового ИИ, но и сравнили 3 наиболее популярных подхода к обучению данных ботов.

\subsubsection{Роль статьи в обзоре}
В статье представлена одна из важнейших концепций в сфере разработки видеоигр. Благодаря ей были созданы игровые боты для многих популярных онлайн игр, и адаптивный игровой ИИ продолжает использоваться в разработке новых видеоигр. Были рассмотрены также наиболее эффективные и оптимальные методы оптимизации обучения такого ИИ, что будет полезно как начинающим разработчикам искуственных интеллектов, так и опытным. Однако разработка и обучение такого ИИ является одновременно еще и одной из самых сложных задач. Для его разработки необходимо создать модель, которая не только работает и принимет решения за наносекунды, но и способна анализировать действия человека разумного. 

\subsection{«Realistic NPCs in video games using different AI approaches», Gustav Grund Pihlgren и другие}
\subsubsection{Краткое описание статьи}
В данной статье более подробно рассматривается обучение ИИ для повышения интеллекта Не-Игровых Персонажей (NPCs). Авторы начинают с рассказа о полезных инструментах, языках программирования и средах разработки (IDE), которые используются для создания игр и обучения ИИ. Затем переходят к самой настоящей разработке игры вместе с читателем статьи, попутно при этом объясняя интересные аспекты и нюансы проектирования таких больших проектов. В конце концов, они переходят к созданию и обучению ИИ для NPC, рассматривая при этом различные подходы обучения ИИ для такой задачи. В заключение же авторы статьи описывают процесс интегрирования такого ИИ в саму игру, а также дают ответы на ряд важных вопросов касательно всего процесса работы с такими играми. 

\subsubsection{Роль статьи в обзоре}
Эта невероятная статья по своей сути является полноценной инструкцией для начинающих в сфере разработки компьютерных игр с использованием ИИ. Авторы описывают весь процесс разработки с нуля, рассказывают о всех возможных нюансах и проблемах, которые только могут возникнуть. В ней подробно рассмотрен такой аспект разработки игр, как работа с NPCs, что очень важно, ибо огромный пласт игр сейчас использует данную механику для улучшения игрового опыта игроков. Помимо этого в статье представлены абсолютно все необходимые инструмены для создания как базовых небольших игровых проектов, так и продвинутых игр с проработанной механикой NPCs. Данная статья собирает в себе все возможные плюсы, и, без сомнения, является одной из полезнейших для начинающих разработчиков. 

\subsection{«3D modelling by means of artificial
intelligence», Berebeshko B и другие}
\subsubsection{Краткое описание статьи}
В данной статье авторы рассмотрели тему создания 3D моделей с использованием ИИ. В статье представлено получение этих моделей из 2D рисунков или картинок. Авторы рассказывают про различные подходы обучения нейросетей, и каждый из подходов уникален, ведь подходит под абсолютно разные задачи 3D моделирования, от создания простых объектов и структур до генерации миров и фрагментов видеороликов. В конце концов авторы рассказывают о имплементации всех вариантов нейросетей на языке программирования c++, анализируют плюсы и минусы данной технологии с учетом потерь качества, сложности дальнейшего использования таких моделей и предпологаемых кадров в секунду при дальнейшем использовании моделей.

\subsubsection{Роль статьи в обзоре}
В статье представлена одна из важнейших для разработки видеоигр концепций, хоть она и не связана с ней прямо. С помощью ИИ для создания 3D моделей можно генерировать внутреигровые объекты, такие как здания, персонажей и оружия, что может очень сильно облегчить работу дизайнеров игр. Кроме того, можно создавать огромные графоподобные структуры из 3D моделей помещений для генерации внутреигровых локаций, таких как подземелья или пещеры. Кроме этого можно добиться генерации целых миров в играх с огромным открытым миром, достаточно лишь иметь какой-либо рандомизирующий алгоритм и варинты 2D картинок для генерации из них уже 3D модели мира. В целом алгоритм, представленный в данной статье, еще не слишком оптимзированный, он теряет большое количество данных и у него есть очень большой шанс генерации с ошибками, однако данный алгоритм все, так или иначе, уже используется в разработке видеоигр дизайнерами для генерации моделей, он имеет большой потенциал развития, и, с большой вероятностью, в будущем он будет являться основой всех глобальных игровых проектов с открытым миром или большим количеством внутреигровых структур.

\subsection{«Text-to-image Diffusion Models in Generative AI:
A Survey», Chenshuang Zhang, Chaoning Zhang, Mengchun Zhang, In So Kweon}
\subsubsection{Краткое описание статьи}
В данной статье представлен еще один базовый концепт, как для разработки видеоигр, так и в целом для дизайна - генерация искуственным интеллектом фотографий из предоставленного текста. Авторы гораздо более подробно раскрывают эту тему. Они рассказывают о базовом представлении модели и принципах ее работы, рассматривают самые популярные и самые оптимизированные модели генерации картинок из текста, такие как GLIDE, Imagen, Stable Diffusion и DALL-E 2. Для продвинутых пользователей они авторы предоставляют огромное количество математических формул и аналитические выкладки, которые позволяют при необходимости реализовать похожую модель самому. Под конец авторы рассматривают возможные улучшения архитектуры моделей, а также дополнительные функции, такие как изменение фотографий с помощью текстовых запросов на изменение и креативная генерация изображений. 

\subsubsection{Роль статьи в обзоре}
Статья была выбрана совсем не просто так. В статье №4 было рассмотрено использование 2D картинок для генерации 3D моделей, которые, в свою очередь, могут быть использованны для создания целых огромных миров или локаций. Тогда же возможность генерировать 2D картинку из текста равна возможности из текста генерировать огромные 3d модели и локации, что является революционным способом создания игр в индустрии компьютерных игр. Помимо такого способа, картинки, генерирумые из текста, могут быть использованы еще во многих аспектах игр. Так, например, ИИ может генерировать загрузочные экраны игр (экран при переходе из локации A в локацию B), а если соединить такой ИИ с адаптивным ИИ, то получится адаптивный генератор загрузочных экранов, который будет генерировать их на основе каких либо выборов игрока по ходу игры. В целом у данной технологии кроме такого еще множество применений, можно смело сказать, что она революционная, и не только в игровой индустрии, именно поэтому данная статья и была внесена в список на обзор.

\subsection{«Video Generation from Text», Yitong Li и другие}
\subsubsection{Краткое описание статьи}
В данной статье описываются возможности генерации видео с использованием ИИ. Авторы статьи рассматривают специально разработанную и обученную модель для генерации видео из текстовых запросов. Они разбирают такие важные понятия, как a Variational Autoencoder (VAE) и Generative Adversarial Network (GAN), это гибридные фреймворки для работы с такими структурами как видео, в которых нужно рассматривать множественные параметры, огромные и очень объемные датасеты, а также постоянно изменяющиеся статические признаки, называемые еще "gist", которые собственно и задают цвет, скорость и изменения на видео (напомню, что стандарт видео сейчас 24 кадра/секунда, что равно примерно 1 кадру в миллисекунду). Также авторы рассказывают, как именно они тренировали данную модель, например c помощью видео с платформы "YouTube", используя сами видео и их названия и описания. В конце концов они предоставляют результаты создания такой модели, возможности ее корректировки, а также дополнительные функции, например изменение и корректировка генерируемового видео прям во время его создания.

\subsubsection{Роль статьи в обзоре}
Эта невероятно интересная статья по сути задает еще один важнейший, но от того и очень сложный концепт - генерацию видеороликов. Как известно из предыдущей статьи, создание 2D картинок сама по себе очень сложная и трудоемкая задача, для нее требуется специально обученная модель, и обучить ее для генерации красивых и соотвествующих запросам картинок очень сложно, для этого требуется долго работать с огромными массивами данных. Сталкиваясь с осознанием всего этого, может появиться мысль, что сгенерировать видео с приемлимым результатом чисто теоретически невозможно, ибо для создания видео скажем в 10 секунд потребуется сгенерировать и правильно склеить около 240 картинок, добиваясь при этом плавности (без потери кадров). Однако авторы данной статьи не только справились со всем этим, но и предложили алгоритм оптимизации и улучшения своей же модели, что кажется невероятным. Таким образом они закрыли огромный пласт проблем, связанных с генерацией внутреигровых роликов, что может быть полезно для создания интерактивых взаимодействий с внутреигровыми персонажами и построения внутресюжетной линии. Статья является очередной инновацией в мире дизайна и разработки игр, и в недалеком будущем механизм, представленный в ней, будет использоватся многими, при этом не только в игровой индустрии

\subsection{«Exploring Controllable Text Generation Techniques», Shrimai Prabhumoye, Alan W Black, Ruslan Salakhutdinov}
\subsubsection{Краткое описание статьи}
В данной статье авторы рассказывают нам о возможности управляемо генерировать текст с использованием специально обученного ИИ. В ней рассматривается разработанная ими модель генерации текста по ключевым словам и запросам. Делят они данную модель на пять модулей, каждый из модулей отвечает за определенный параметр генерации модели, соотвественно для контроля и изменения параметров генерации необходимо модифицировать эти модули. Кроме того авторы рассматривают техники разработки, контроля и управления такими модулями для грамотной работы с моделью. В статье также были рассмотрены дальнейшие методы разработки новейших архитектур, моделей, построенных на представленных модулях. В конце же авторы предоставляют плюсы и минусы использования такой модульной модели ИИ, рассказывают о возможностях улучшения и оптимизации данных модулей.

\subsubsection{Роль статьи в обзоре}
Статья открывает глаза на еще одну важную концепцию, которая является одной из ключевых как в разработке игр, так и в других сферах, таких как написание книг, создание сюжетов для фильмов, дизайна и многих других. Управляемая генерация текста является безумно полезным инструментов, так как дополняет концепции, которые мы уже рассмотрели, такие как генерация картинок и видео из текста. У дизайнера игр появляется возможность сгенерировать любую инструкцию по генерации картинок или видео с использованием только лишь ключевых слов, что может сильно облегчить его работу. Более того, в больших игровых проектах есть специальные люди, которые прописывают сюжет и сюжетные фразы для внутреигровых персонажей. С помощью управляемой генерации текста можно создавать сюжеты, задавая при этом только ключевые слова и направление создания сюжета. В общем и целом такие модели могут очень сильно облегчить жизнь разработчиков, а данная статья представляет собой подробный гайд, как создать модель под необходимую именно разработчику задачу, что делает эту статью безумно полезной в разработке игр.

\subsection{«Generating Music using AI», Ebba Rickard}
\subsubsection{Краткое описание статьи}
Данная статья представляет собой обзор на еще один способ применения искусственного интеллекта - генерация музыки из текста.
Исследование фокусировалось на возможностях создания лицензионно-свободной фоновой музыки при помощи машинного обучения. Автором были проанализированы существующие модели и данные, а также она проведела интервью с музыкантами для выявления характеристик качества музыки. В работе были выбраны и сравнены модели GPT-3 и Performance RNN, а оценка музыкальных композиций выполнялась с использованием алгоритма COSIATEC и специальной музыкальной метрики. Проведенные эксперименты включали в себя изучение воздействия параметров обучения и характеристик предоставленных данных. Обе модели успешно демонстрировали способность к генерации долгосрочной структуры в музыке, но с различным временем обучения и точностью в зависимости от выбора данных. Кроме того, автор изучила, как связана информация, полученная из исследования, и реальное восприятие человеком музыки.

\subsubsection{Роль статьи в обзоре}
Статья дополняет все возможности использования ИИ для генерации чего либо. В данном случае это была музыка, и, на самом деле, эта задача является еще более сложной, чем генерация видео. Видео состоит из картинок, тогда как музыка - композиция, использующая бесконечное количество различных инструментов и их производных, звуков. Для генерации музыки нужно очень хорошо понимать то, как обычный человек воспринимает музыку, то, по каким критериям строятся и создаются песни, и даже понимая это, все равно необходимо создать модель, которая будет работать с такими критериями как с параметрами и правильно и грамотно составлять музыку. Данная статья без сомнения является очень важной, ибо предоставленная в ней информация может быть революционной для мира музыки и безумно полезной для индустрии разработки компьютерных игр.

\subsection{«Automated AI Planning and Code Pattern Based Code Synthesis», Jicheng Fu, Farokh B, Bastani, I-ling Yen}
\subsubsection{Краткое описание статьи}
В данной статье был рассмотрен ИИ для автоматической генерации и анализа уже написанного кода. Ввиду все большего усложнения кода разработки, человеческий мозг очень часто не справляется с такими большими проектами, именно поэтому автор рассмотрела такой ИИ. Она рассказывает о том, что текущие ИИ не способны генерировать комплексно сложные циклы и функции в коде, или в целом работать с низкоуровневым кодом (работа с памятью, процессором, видеокартой). В своей статье она разбирает методы улучшения существующих ИИ до ИИ с возможностью написания таких структур кода, разбирает такие базовые примитивы, как Code Pattern Integration System (CPIS), что является механизмом интеграции в код для полной его автоматизации, а также с Component Based Software Development (CBSD), т.е. с разработкой кода "кусками", большими блоками.

\subsubsection{Роль статьи в обзоре}
Методы, представленные в данной статье, могут помочь уже разработчикам самого кода игры. В статье представлен метод автоматической генерации кода, что очень полезно в любой разработке игр. Система будет автоматически дописывать и анализировать код, и это поможет любому разработчику лучше разобраться в коде. Метод, представленный в статье, пока сыроват и требует доработок, однако многие компании, например JetBrains, уже используют подобные системы в своих средах разработки, делая жизнь разработчиков намного проще. Для тех же, кому нужен специализированный ИИ под его конкретные нужды, и существует данная статья, в которой представлен метод создания такого ИИ

\subsection{«Asyncflow: A visual programming tool for game artificial intelligence», Zhipeng Hu и другие}
\subsubsection{Краткое описание статьи}
В данной статье авторы представляют важный вклад в область разработки игр с помощью своей статьи о Asyncflow - открытом визуальном программном инструменте, призванном улучшить эффективность создания искусственного интеллекта и геймплея. В статье представлены инструкции по создателю блок-схем для пояснения логики игры и среды выполнения с асинхронным механизмом на основе событийно-управляемой архитектуры. Ключевое преимущество этого инструмента - возможность работать с любым языком програмирования, будь то C#, C++ или Java. Также они рассказывают о том, что данный инсрумент работает с любым игровым движком, что делает его просто незаменимым для разработки игр любого жанра. В конце же они рассказывают о грамотном внедрении данного инструмента в среды разработки и поясняют, каким образом это наиболее правильно сделать.

\subsubsection{Роль статьи в обзоре}
Статья по сути рассказывает о том, что в современном мире все больше и больше разработчиков пытаются внедрить ИИ для того, чтобы облегчить свою жизнь. Инструмент из статьи является инновационным продуктом, и он может помочь любому разработчику как начать разработку своего ИИ, так и улучшить уже имеющися, или подстроить ИИ из открытых источников под свои нужды. Статья полезна как начинающим разработчикам, так и продвинутым пользователям, она может сделать их жизнь чуть-чуть проще и продуктивнее.


\newpage
\section{Список литературы и ссылки на статьи}
\begin{itemize}
 \item \href{https://www.researchgate.net/publication/375488989_The_Role_of_Artificial_Intelligence_in_Video_Game_Development}{«The Role of Artificial Intelligence in Video Game Development», Aleksandar Filipovic}
 \item \href{https://www.sciencedirect.com/science/article/pii/S0167642307000548}{«Knowledge acquisition for adaptive game AI, Marc Ponsen», Pieter Spronck, Héctor Muñoz-Avila, David W. Aha} 
 \item \href{https://odr.chalmers.se/items/f2c23a1d-b9a9-4148-b92f-183907970e2d}{«Realistic NPCs in video games using different AI approaches», Gustav Grund Pihlgren и другие}
 \item \href{https://www.jatit.org/volumes/Vol99No6/5Vol99No6.pdf}{«3D modelling by means of artificial intelligence», Berebeshko B и другие}
 \item \href{https://arxiv.org/abs/2303.07909}{«Text-to-image Diffusion Models in Generative AI: A Survey», Chenshuang Zhang, Chaoning Zhang, Mengchun Zhang, In So Kweon}
 \item \href{https://ojs.aaai.org/index.php/AAAI/article/view/12233}{«Video Generation from Text», Yitong Li и другие}
 \item \href{https://arxiv.org/abs/2005.01822}{«Exploring Controllable Text Generation Techniques», Shrimai Prabhumoye, Alan W Black, Ruslan Salakhutdinov}
 \item \href{https://lup.lub.lu.se/luur/download?func=downloadFile&recordOId=9093922&fileOId=9093927}{«Generating Music using AI», Ebba Rickard}
 \item \href{https://ieeexplore.ieee.org/abstract/document/4031942?casa_token=WF7oQXaCYSUAAAAA:xLmkqoGgGjGWYHPN3cZaEISoyrqlLnhMD1CdkVGuVNtLOJLL1fdwSEyDPw5AcKUsFKoIep8Jx_bkEQ}{«Automated AI Planning and Code Pattern Based Code Synthesis», Jicheng Fu, Farokh B, Bastani, I-ling Yen}
 \item \href{https://www.sciencedirect.com/science/article/pii/S2468502X21000498}{«Asyncflow: A visual programming tool for game artificial intelligence», Zhipeng Hu и другие}
\end{itemize}

\end{document}